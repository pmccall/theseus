%
% File: thesisexample.tex   Version 1.8   May 6, 1999
%
% This is an example document for easychithesis, a 
% LaTeX2e style for formating theses at the University of
% Chicago.The easychithesis style was written by Bryan Clair and 
% Nathan Dunfield.  It contains detailed instructions for using
% easychithesis.
%
%  HINTS:
%
%  1. To get appendices, you don't do anything different from a normal
%     report document.  That means, put the command \appendix before
%     you begin your first appendix, then do each appendix with a
%     \chapter command.  Note that if you have only one appendix, it is
%     customary to leave it unnumbered.  Do this with \chapter*.
%
%  2.  If you use \chapter*, which produces unnumbered chapters, you 
%       have to add that chapter to the table of contents by hand, e.g.
%
%         \chapter*{Appendix}
%         \addcontentsline{toc}{chapter}{Appendix}
%
%  3.  If you get errors about
%   
%    Undefined control sequence.
%         \setspace@size ...rrsize \normalsize \@normalsize \else \@currsize \fi 
%
%     You need a newer version of the file ``setspace.sty''.
%
%  4.  Problems with math formulas in chapter headings:
% 
%         a.  Any lowercase letters in the formula are converted to
%         uppercase, e.g. f(x) becomes F(X).   If you really need
%         lowercase math letters in your chapter titles, use the
%         option plainchapterheads (and, if you want, type your
%         chapter titles in ALL CAPS so that the appearance doesn't
%         change).  Note there is no problem 
%         for section or subsection headings in either case.  (Options
%         such as plainchapterheads are given as part of the 
%         \documentclass command, see below under ``Document Options'').
%
%         b.  Some perfectly reasonable math commands when used in
%         \chapter give the error
%          ``LATEX ERROR: \command  ALLOWED ONLY IN MATH MODE.''
%         The solution to this is to do
%        
%              \newcommand{\mymath}{problem math goes here}
%       
%         and then
%        
%              \chapter{All about \protect\mymath}
%
%         also, the option plainchapterheads will fix this too.
%
%  5. If your References section doesn't show up in the table of
%     contents, you need to add the line \addcontentsline...
%     as done at end of this file.   Make sure that you include the 
%     page break command as done there or else you may end up
%     with the wrong page number in your table of contents. 


\documentclass{easychithesis}
% Document Options: 
%
% Note if you want to save paper when printing drafts, replace the
% above line by
% 
%   \documentclass[singlespace]{easychithesis}
% 
% or 
% 
%   \documentclass[onehalfspace]{easychithesis}
%
% Also the ``double spacing'' provided by this style is not ``true''
% doublespacing as defined by setspace.sty.  Instead, it is the same
% as on the old LaTeX 2.09 thesis style ``chithesis''.  If you want 
% ``true'' doublespacing (if there is such a thing), give the option 
% truedoublespace.  The increase in tree murder will be on your 
% conscience, not mine.
%
%  Similarly, if you need to use the plainchapterheads option, you do
%   
%  \documentclass[plainchapterheads]{easychithesis}
%
% You can give more than one option, if you desire.

\begin{document}

% Create the official title
\title{Big Ass Waste of Time} 
\author{Will McFadden}
\date{Sept 2016}
\department{Biophysical Science}
\division{Physical Sciences} 
\degree{Doctor  of Philosophy} 
\maketitle

% \dedication : Use for a dedication, copyright, or epigraph.
%               Produces a page with no number for the text which follows
%               If you want centering, do it yourself with 
%               \begin{center} and \end{center}.  You can have more
%               than one `dedication'.
\dedication
\begin{center}
        To 
\end{center}


% \topmatter : Things like Abstract, Acknowledgements.
% For the abstract, you can also do 
%      \begin{abstract} ...text... \end{abstract}
% if you prefer.

\topmatter{Abstract}
This paper concerns the ridiculous Yes song, ``Cans And Brahms''.
We investigate what could possibly have prompted Rick Wakeman to
cover the classic Brahms piano piece with his cheesy Moog synth.
We explore the relationship between his solo and the other solo
endeavors on the \emph{Fragile} album.

\topmatter{Acknowledgements}
I'd like to thank Jason Bermack, who unknowingly played
\emph{Big Generator} for me in his car, when we were both in high 
school, thus ensuring a long and fruitful relationship with Yes music 
throughout my college days.

I'd also like to thank Dave Moulton, who likes Yes too, and 
accompanied me to my first ever rock concert - Yes in the Oakland 
Coliseum.  They played ``Awaken'' for 20 minutes.  Wow.

%
% Table Of Contents
%

\tableofcontents

%
% List of figures
% 

\listoffigures

% 
% List of tables
% 

\listoftables

%
% Begin Body
%
\mainmatter

%
% Body Chapters
%
\chapter{Introduction}


\chapter{Main Work}

\chapter{Other Solos on Fragile}

\chapter{Yes Excesses Over Time}

%
% Appendices
%
\appendix
\chapter{Incomplete Discography I}
Here's a table of some Yes Albums I can think of, in something that's 
close to but certainly not in chronological order.
\begin{table}[h]
\begin{center}
\begin{tabular}{|c|}\hline
        Time And A Word\\
        \hline
        Fragile\\
        \hline
        The Yes Album\\
        \hline 
        Close To The Edge\\
        \hline 
        Tales From Topographic Oceans\\
        \hline
        Drama\\
        \hline
        Tormato\\
        \hline 
        Relayer\\
        \hline
        Going For The One\\
        \hline
        90125\\
        \hline 
        Big Generator\\
        \hline
        Union\\
        \hline 
        Talk\\
        \hline
\end{tabular}
\caption{Some Yes Albums}
\end{center}
\end{table}

\chapter{Yes Members}
Of course, nobody can keep track of the various members of the band 
over the years, though Jon and Chris form the backbone and I believe 
have appeared on every album.  Except maybe Chris missed one in there 
somewhere.  Anyway, here's some of the other members, with their main 
instruments.  Pretty much everyone (but Bill?) sings at some point, if 
only for harmony.
\begin{table}[h]
\begin{center}
\begin{tabular}{|c|c|}
        \hline
        Jon Anderson & vocals \\
        Bill Bruford & drums \\
        Steve Howe & guitar\\
        Tony Kaye & keyboard\\
        Trevor Rabin & guitar\\
        Rick Wakeman & keyboard\\
        Alan White &drums\\
        \hline
\end{tabular}
\caption{Yes Members}
\end{center}
\end{table}

\chapter{Useless figures}
Here's a pointless figure:
\begin{figure}[htbp]
  \begin{center}
    {\Huge Yes!}
    \caption{Yes}
  \end{center}
\end{figure}
and another:
\begin{figure}[htbp]
   \begin{center}
     {\Huge No!}
     \caption{No}
    \end{center}
\end{figure}

%
% References (the thesis office prefers that to Bibliography...)
%
% They also prefer that it be single spaced.
% The pagebreak command is necessary inorder to insure that 
%  the page number that appears in the table of contents is
%  the correct one.   

\singlespacing
\pagebreak
\addcontentsline{toc}{chapter}{References}

\begin{thebibliography}{99}
   % the 99 is as wide or wider than any bibliography labels.
\bibitem{fragile}Anderson, Bruford, Squire, Wakeman.  \emph{Fragile}.
  Atlantic, 1973.
\end{thebibliography}

\end{document}