%% outline-sample.tex
%% Copyright 1991 Peter Halvorson
%% Updates for LaTeX2e copyright 2002 Seth Flaxman
%% Updated for LPPL 1.3c or later by Clea F. Rees (for Seth Flaxman), 2008/10/06.
%
% This work may be distributed and/or modified under the
% conditions of the LaTeX Project Public License, either version 1.3
% of this license or (at your option) any later version.
% The latest version of this license is in
%   http://www.latex-project.org/lppl.txt
% and version 1.3 or later is part of all distributions of LaTeX
% version 2005/12/01 or later.
%
% This work has the LPPL maintenance status `unmaintained'.
%
% This work consists of the files outline.sty and outline-sample.tex.
% Save file as: outline-sample.tex

\documentclass{report}
\usepackage{outline}

% [outline] includes new outline environment. I. A. 1. a. (1) (a)
% use \begin{outline} \item ... \end{outline}

\pagestyle{empty}

\begin{document}

\begin{outline}
  \item {\bf Introduction }
  \begin{outline}
    \item {\bf Relevance } \\
      Describe how essential actin networks are to the deformability of cells
    \item {\bf Biology } \\
      Describe particulars of actin networks in detail.  Arrive at conclusion that tradeoffs between stress generation and relaxation are important.
    \item {\bf Prior Work } \\
      Reveal the difficulties addressed in microscopic models of the past. 
    \item {\bf Our Work } \\
      Describe our novel contributions and how they reveal new insights.
  \end{outline}
  \item {\bf Results}
  \begin{outline}
    \item {\bf Figures } 
      \begin{outline}
      \item {\bf Graphical Abstract } \\
	Describe the idea of having to balance force generation and stress relaxation and how recycling let's you tune them separately.
      \item {\bf Model Description } \\
	Point out the individual features of actin networks and how we incorporate them into this model
      \begin{outline}
	\item {\bf Assymetric springs } 
	\item {\bf Cross-link slip } 
	\item {\bf Fractional myosin activity } 
	\item {\bf Filament recycling } 
	\item {\bf Computational techniques } 
      \end{outline}
      \item {\bf Effective Viscosity } \\
	Creep deformation and effective viscosity of filament networks in the presence of cross-link slip.
      \begin{outline}
	\item {\bf Simulation } \\
	   Before and after application of extensional stress to a cross linked network.
	\item {\bf Strain measurement } \\
	   Quantification of material strain rates for simulation in panel a
	\item {\bf Dependence of effective viscosity on network architecture } \\
	   Effective viscosity depends on $L/l_c$ 
	\item {\bf Dependence of relaxation time on filament and cross-link properties } \\
	   The time to reach steady state strain is dependent on the filament extensional modulus and the cross-link slip rate.
      \end{outline}
      \item {\bf Active Stress Buildup } \\
	Transient contraction and stress buildup in networks of asymmetric springs.
      \begin{outline}
	\item {\bf Simulation } \\
	   Before and after application of active stress to a cross linked network.
	\item {\bf Stress measurement } \\
	   Quantification of material strain rates for simulation in panel a
	\item {\bf Dependence of maximum stress on network architecture } \\
	   The maximum stress achieved depends on $L/l_c$ 
	\item {\bf Dependence of maximums stress time on filament and cross-link properties } \\
	   The time to reach the peak stress is dependent on the filament extensional modulus and the cross-link slip rate.
      \end{outline}
      \item {\bf Stress and Viscosity from Filament Recycling} \\
	Steady-state creep deformation and stress buildup in networks with filament recycling
      \begin{outline}
	\item {\bf Strain measurements } \\
	   Quantification of material strain rates for simulation in which filaments are being recycled.
	\item {\bf Dependence of effective viscosity on recycling rate } 
	\item {\bf Stress measurements } \\
	   Quantification of stress buildup and dissipation for simulation in which filaments are being recycled.
	\item {\bf Dependence of steady state stress on recycling rate } 	
      \end{outline}
      \item {\bf Flows in Recycling Networks } \\
	Bring it all home somehow.
    \end{outline}
  \end{outline}
\end{outline}

\end{document}
